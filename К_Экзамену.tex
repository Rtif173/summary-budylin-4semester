\documentclass{article}
\usepackage{cmap}
\usepackage[utf8]{inputenc}
\usepackage[russian]{babel}
\usepackage[14pt]{extsizes}
\usepackage{amsmath, amsfonts, amssymb}
\usepackage[left=1.5cm,right=1.5cm,top=1.5cm,bottom=1.5cm,bindingoffset=0cm]{geometry}

\newcommand{\Lg}{Лагранжа}
\newcommand{\Eu}{Эйлера }
\newcommand{\EL}{Эйлера-Лагранжа}
\begin{document}

\section{Ряды и интегралы Фурье. Интеграл Фурье.}

\subsection{Что такое преобразование Фурье абсолютно интегрируемой функции? Опишите его основные свойства.}

Для абсолютно интегрируемой функции $f(x)$ справедливо равенство
\[ f(x)=\frac{1}{2\pi}\int_{-\infty}^{\infty}{ d\xi \int_{-\infty}^{\infty} f(t)e^{i\xi(x-t)}dt}\]
Правая часть равенства называется интегралом Фурье.\\
Это выражение можно разделить на две части. Например, в симметричном варианте
\[ \widehat{f}(\xi) = \frac{1}{\sqrt{2\pi}}\int_{-\infty}^{\infty}f(x)e^{-i\xi x}dx  \]
\[ f(x) = \frac{1}{\sqrt{2\pi}}\int_{-\infty}^{\infty}\widehat{f}(\xi)e^{i\xi x}d\xi  \]
Преобразованием Фурье функции $f$ называется функция $\widehat{f}$

Преобразованием (или оператором) Фурье называется отображение
\[ F: f \rightarrow \widehat{f}=Ff, \widehat{f}(\xi) = \frac{1}{\sqrt{2\pi}}\int_{-\infty}^{\infty}f(x)e^{-i\xi x}dx\]

Свойства преобразования Фурье функции $f$:
\begin{enumerate}
	\item $ \widehat{f}(\xi) $ -- отграниченная функция
	\item $ \widehat{f}(\xi) $ -- равномерно непрерывная функция
	\item $ \widehat{f}(\xi) \underset{\xi\rightarrow \infty}\longrightarrow 0$
\end{enumerate}

\subsection{Сформулируйте теорему об обращении преобразования Фурье.}
Если $ f $ -- абсолютно интегрируемая функция, то имеет место формула обращения (обратное преобразование Фурье)
\[ f(x) = \frac{1}{\sqrt{2\pi}}\int_{-\infty}^{\infty}\widehat{f}(\xi)e^{i\xi x}d\xi  \]

\subsection{Что представляет собой равенство Парсеваля для преобразования Фурье? Дайте примеры.}
Равенство Парсеваля состоит в том, что 
\[ \int_{-\infty}^{\infty} \left| \widehat{f}(\xi) \right| ^2 d\xi =  \int_{-\infty}^{\infty} \left| f(x) \right| ^2 dx \]
Для равенства Парсеваля достаточно квадратичной интегрируемости функций
%todo Примеры

\subsection{Как найти преобразование Фурье производной? Какова связь между гладкостью функции и скоростью убывания ее преобразования Фурье на бесконечности?}
\[ F\left(\frac{d^n f}{dx^n}\right)(\xi)=(i\xi)^n \widehat{f}(\xi) \]

\subsection{Как найти преобразование Фурье произведения функции на аргумент? Какова связь между скоростью убывания функции на бесконечности с гладкостью ее преобразования Фурье?}
\[ F\left(x^n f\right)(\xi)=i^n \frac{d^n \widehat{f}(\xi)}{d\xi^n} \]

\subsection{Дайте определение свёртки функций на оси. Чему равно преобразование Фурье свертки.}
Свёрткой на оси называется интеграл
\[ f * g = \int_{-\infty}^{\infty} f(t)g(x-t)dt \]

Если $ f $ и $ g $ -- непрерывны, абсолютно и квадратично интегрируемы, то свёртка $ f*g $ абсолютно интегрируема и
\[ \widehat{f*g} = \widehat{f} \cdot \widehat{g} \]

\subsection{Что такое гауссова плотность? Чему равно преобразование Фурье гауссовой плотности?}

\subsection{Асимптотика состояния свободной нерялитивисткой квантовой частицы на больших временах.}


\section{Элементы геометрии поверхностей}

\section{Вариационное исчисление. Простейшие задачи вариационного исчисления}

\subsection{Что такое интегральный функционал и его производная? Что такое вариация интегрального функционала?}
Интегральный функционал -- отображение
\[I: y \rightarrow I(y)=\int_{x_1}^{x_2} F(x, y, y')dx \]

Производная интегрального функционала называется вариацией
\[ \delta I[\eta]= \left. \frac{d}{dt}I(y+t\eta) \right|_{t=0} \]

\subsection{Сформулируйте основную лемму вариационного исчисления. Опишите идею ее доказательства.}
Пусть $ G(x) \in C[x_1, x_2] $, и для любой $ \eta $, такой что $ \eta(x) \in C^{(1)}[x_1, x_2] $, $ \eta(x_1)=\eta(x_2)=0 $
выполнено \[ \int_{x_1}^{x_2} \eta(x)G(x)dx = 0 \]
то $ G(x) \equiv 0 $ на интервале $ [x_1, x_2] $

Доказательство ведётся от противного. \\
Пусть $ \exists x_0 \in (x_1, x_2): G(x_0) > 0 $. Тогда в силу непрерывности $ G(x) $ 
$ \exists(a,b)\subset(x_1, x_2): \forall  x \in (a,b) \quad G(x>0) $

Зададим $ \eta $ следующим образом
\[\eta = \left\{
\begin{matrix}
	(x-a)^2(x-b)^2 	& x\in(a,b) \\ 
	0				& x\notin(a,b)
\end{matrix}
\right.
\]
$ \eta $ имеет непрерывную производную и $ \eta(x_1)=\eta(x_2)=0 $

Но 
\[ \int_{x_1}^{x_2} \eta(x)G(x)dx=\int_{a}^{b}(x-a)^2(x-b)^2G(x)dx>0 \]
Получили противоречие 
\subsection{Что такое уравнение Эйлера-Лагранжа? Как оно получается?}
Пусть $y(x)$ минимизирует интеграл 
\[ I = \int_{x_1}^{x_2}F(x, y, y')dx \]
и $ y(x_1)=y_1, y(x_2)=y_2 $

Пусть $Y(x)=y(x)+t\eta(x)$, где $ \eta(x_1)=\eta(x_2)=0 $. \\
Заменим в $I$ $y(x)$ на $Y(x)$ и $y'(x)$ на $Y'(x)$ и зафиксируем функцию $ \eta(x) $. Тогда
\[ I(t) = \int_{x_1}^{x_2}F(x, Y, Y')dx \]
Задача свелась к нахождению минимума $ I(t) $. Минимум достигается при  $y(x)=Y(x)$, т.е. при $ t=0 $. 
Таким образом получаем, что $ I'(0)=0 $
\[ \frac{dI}{dt} = \int_{x_1}^{x_2}\frac{\partial F}{\partial Y} \eta + \frac{\partial F}{\partial Y'}\eta' dx \]
Воспользуемся тем, что $ I'(0)=0 $
\[ \int_{x_1}^{x_2}\frac{\partial F}{\partial y} \eta + \frac{\partial F}{\partial y'}\eta' dx = 0 \]
Проинтегрировав второе слагаемое по частям получим
\[ \int_{x_1}^{x_2}  \left( \frac{\partial F}{\partial y} - \frac{d}{dx}\frac{\partial F}{\partial y'}\right)\eta dx = 0 \]
В силу основной леммы 
\[  \frac{\partial F}{\partial y} - \frac{d}{dx}\frac{\partial F}{\partial y'} =0 \text{ -- Уравнение Эйлера-Лагранжа} \]

\subsection{Что такое экстремали? Что называется вариационной производной? Почему?}
Экстремалями называются решения уравнения \EL 

Рассмотрим первую вариацию функционала
\[ I'(0) = \delta I [\eta] = \int_{x_1}^{x_2}  \left( \frac{\partial F}{\partial y} - \frac{d}{dx}\frac{\partial F}{\partial y'}\right)\eta dx \]
пологая $ \eta = \delta y $
\[ \delta I = \int_{x_1}^{x_2}  \left( \frac{\partial F}{\partial y} - \frac{d}{dx}\frac{\partial F}{\partial y'}\right)\delta y dx \]
$ \delta I$ линейно зависит от $ \delta y $. По аналогии с производной по Фрише вариационной производной функционала $I$ называют
\[ \frac{\delta I}{\delta y} = \frac{\partial F}{\partial y} - \frac{d}{dx}\frac{\partial F}{\partial y'} \]

\subsection{Пусть функция Лагранжа $F(x, y, y')$ не зависит от переменной $x$. Что представляет собой первый интеграл уравнения Эйлера-Лагранжа? Почему?}
$F=F(y, y')$
\[ \frac{d}{dx}\left ( y'\frac{\partial F}{\partial y'} - F \right ) =  y''\frac{\partial F}{\partial y'}+y'\frac{d}{dx}\frac{\partial F}{\partial y'} - \frac{\partial F}{\partial x} - \frac{\partial F}{\partial y}y'-\frac{\partial F}{\partial y'}y'' = -y'\left(\frac{\partial F}{\partial y}- \frac{d}{dx} \frac{\partial F}{\partial y'} \right) \]
В случае экстремали предыдущее выражение равно 0, значит
\[ \frac{d}{dx}\left ( y'\frac{\partial F}{\partial y'} - F \right ) =0 \]
\[  y'\frac{\partial F}{\partial y'} - F  = C  \]
Предыдущее равенство называется первым интегралом уравнения \EL

\subsection{Что такое геодезические? Как ставится соответствующая вариационная задача? Как она решается в случае ортогональных локальных координат при условии, что коэффициенты первой квадратичной формы зависят лишь от одной из переменных?}
Пусть поверхность задана уравнением $ g(x,y,z)=0 $.\\
Кривая минимальной длины, соединяющая заданные точки называется геодезической

Пусть элемент поверхности $ dS = P(u,v)(du)^2 + 2Q(u,v)dudv+R(u,v) (dv)^2 $. \\
Зафиксируем на поверхности точки $ P_1(u_1, v_1) $ и $ P_2(u_2, v_2) $. 

Рассмотрим кривую $ v = v(u) $. Её длина 
\[ I = \int_{u_1}^{u_2} \sqrt{P+2Qv'+Rv'^2}du \]
При $ Q=0$ (условие ортогональности), $P=P(v)$, $R=R(v) $ функция \Lg{} не зависит от $u$
\[ v'\frac{Rv'}{\sqrt{Rv'^2 + P}}-\sqrt{Rv'^2 + P}=C \]
\[ v=C\int \frac{\sqrt{R}dv}{\sqrt{P^2 -C^2P}} \]

\subsection{Что такое естественные граничные условия? В каких вариационных задачах они возникают?}
Задача со свободными концами: найти экстремали \[ I = \int_{x_1}^{x_2}F(x,y,y')dx \] удовлетворяющие граничному условию $y(x_1)=y_1$

Для такой задачи естественные граничные условия на правом конце
\[ \left. \frac{\partial F}{\partial y} \right|_{x=x_2} = 0 \]


\section{Вариационное исчисление. Обобщения, приложения и трансверсальность}
\subsection{Выпишите уравнения Эйлера-Лагранжа в случае нескольких функций. Как они получаются?}
Пусть 
\[ I=\int_{x_1}^{x_2} F(x, y,..., z, y', ..., z')dx \]
минимизируют функции $y, ..., z$\\
Введём 
\[ Y=y+t\eta, ..., Z=z+t\xi \]
и заменим в $I$ $y$ на $Y$,..., $z$ на $Z$, тем самым сведём задачу к поиску минимума функции $ I(t) $ одной переменной. 
Минимум достигается при $ I'(0)=0 $.

Далее, аналогично случаю с одной функцией, с учётом того, что $\eta, ..., \xi$ независимы, 
положим все переменные кроме $\eta$ равными нулю. Получим, что
\[ \frac{\partial F}{\partial y} - \frac{d}{dx}\left(\frac{\partial F}{\partial y'}\right)=0 \] 
Аналогично для остальных функций.

В итоге придём к системе дифференциальных уравнений 
\[ \left\{
	\begin{matrix}
		\frac{\partial F}{\partial y} - \frac{d}{dx}\left(\frac{\partial F}{\partial y'}\right)=0\\ 
		\vdots \\
		\frac{\partial F}{\partial z} - \frac{d}{dx}\left(\frac{\partial F}{\partial z'}\right)=0
	\end{matrix}
\right. \]

\subsection{Сформулируйте принцип наименьшего действия в лагранжевой механике. Что представляют собой уравнения Эйлера -Лагранжа, если кинетическая энергия равна $ T=\frac{m(x'^2+y'^2+z'^2)}{2} $ , а потенциальная $U=U(x, y, z)$?}
Рассмотрим систему $ N $ частиц. 
Кинетическая энергия \[ T=\sum_{i=1}^{N}\frac{m_i(\dot{x_i}^2+\dot{y_i}^2+\dot{z_i}^2)}{2} \]
Потенциальная \[ U=U(x_1,..., z_n) \]
Пусть система может находиться лишь на некоторой фиксированной поверхности, заданной $ l $ голономными связями. 
Введём на этой поверхности локальные координаты $q_1, ..., q_{k=3N-l}$, называемых обобщёными. 
Тогда $ T = T(q_1,..., q_k, \dot{q_1}, ..., \dot{q_k})$ и $ U=U(q_1, ..., q_k) $.

Принцип наименьшего действия (принцип Гамильтона): действительным положением системы на временном интервале $ [t_1, t_2] $ 
является экстремаль функционала \[ I=\int_{t_1}^{t_2} L(q_1,...,q_k,\dot{q_1}, ..., \dot{q_k})dt, \]
где $L=T-U$

Так как $L$ не зависит от времени явно можно воспользоваться первым интегралом уравнения Эйлера-Лагранжа. 
Получим закон сохранения полной энергии $T+U=\text{const}$

\subsection{Выпишите уравнение Эйлера-Лагранжа для экстремалей двойных интегралов. Как оно выводится?}

\[ I=\iint_D F(x, y, z, z'_x, z'_y)dxdy \]
Пусть $ z(x,y) $ минимизирует интеграл $ I $.\\
$ Z(x,y)=z(x,y) + t\xi(x,y), \qquad \left. \xi(x,y) \right| _{\partial D} = 0 $\\
Заменим $ z $ на $ Z $ в $ I $, условие экстремальности $ I'(0)=0 $
\[ I'(0)=\iint_D\left(\frac{\partial F}{\partial z}\xi + \frac{\partial F}{\partial z'_x}\xi'_x+\frac{\partial F}{\partial z'_y}\xi'_y\right)dxdy=0 \]
Рассмотрим интеграл 
\[ \iint_{D}\left ( \frac{\partial }{\partial x} \left ( \frac{\partial F}{\partial z'_x}\xi \right )+\frac{\partial }{\partial y} \left ( \frac{\partial F}{\partial y'_x}\xi \right )\right )dxdy \]
С одной стороны \[ \iint_{D}\left ( \frac{\partial }{\partial x} \left ( \frac{\partial F}{\partial z'_x}\xi \right )+\frac{\partial }{\partial y} \left ( \frac{\partial F}{\partial y'_x}\xi \right )\right )dxdy=
\oint _{\partial D} \left ( \frac{\partial F}{\partial z'_x}\xi dy- \frac{\partial F}{\partial y'_x}\xi dx \right )=0 \]
В предыдущем выражении воспользовались формулой Грина и нулевыми граничными условиям $ \xi $. С другой
\[ \begin{split}
\iint_{D}&\left ( \frac{\partial }{\partial x} \left ( \frac{\partial F}{\partial z'_x}\xi \right )+
\frac{\partial }{\partial y} \left ( \frac{\partial F}{\partial y'_x}\xi \right )\right )dxdy
= \\&=
\iint_{D}\left ( \frac{\partial }{\partial x}\frac{\partial F}{\partial z'_x}+\frac{\partial }{\partial y} \frac{\partial F}{\partial y'_x}\right )\xi dxdy+
\iint_{D}\left (\frac{\partial F}{\partial z'_x}\xi'_x+\frac{\partial F}{\partial z'_y}\xi'_y\right )dxdy 
\end{split}
\]
Таким образом получим 
\[ \iint_{D}\left (\frac{\partial F}{\partial z}- \frac{\partial }{\partial x}\frac{\partial F}{\partial z'_x}-\frac{\partial }{\partial y} \frac{\partial F}{\partial y'_x}\right )\xi dxdy =0 \]
В силу основной леммы 
\[ \frac{\partial F}{\partial z}- \frac{\partial }{\partial x}\frac{\partial F}{\partial z'_x}-\frac{\partial }{\partial y} \frac{\partial F}{\partial y'_x} = 0 \]
\subsection{Что представляет собой волновое уравнение? Дайте его интерпретацию как уравнения Эйлера-Остроградского.}
Пусть струна плотности $\rho$ натянута вдоль оси $x$ (натяжение $\tau$) и совершает колебания в плоскости $xu$ 
перпендикулярно оси $x$. Тогда колебания $u(x,t)$ струны удовлетворяют волновому уравнению
\[ \frac{\partial^2 u}{\partial t^2} = a^2 \frac{\partial^2 u}{\partial x^2}, a=\sqrt{\frac{\tau}{\rho}} \] 
%TODO Дайте его интерпретацию как уравнения Эйлера-Остроградского.

\subsection{Как ставится изопериметрическая задача? Как она сводится к задаче на условный экстремум функции нескольких переменных?}
Изопериметрическая задача ставится ставится следующим образом. Пусть кривая $ y=y(x) $ с фиксированными концами $ y(x_{1,2})=y_{1,2} $
минимизирует интеграл
\[ I=\int_{x_1}^{x_2} F(x, y, y')dx \]
причём интеграл 
\[ J=\int_{x_1}^{x_2} G(x, y, y')dx \]
обладает заранее заданным значением. Какому дифференциальному уравнению должна удовлетворять $y$?
%TODO Как она сводится к задаче на условный экстремум функции нескольких переменных?
 
\subsection{Как ставится задача Лагранжа? В чем отличие голономных связей от неголономных? Что такое множители Лагранжа в случае задачи Лагранжа, в чем их отличие от множителей Лагранжа для изопериметрической задачи?}
Задача Лагранжа ставится следующим образом. Найти минимизирующие функции $y_1, ..., y_n$ интеграла
\[ I=\int_{x_1}^{x_2} F(x, y_1, ..., y_n, y_1', ..., y_n')dx \]
при условии, что 
\[ \left \{
\begin{matrix}
	G_1(x, y_1, ..., y_n, y_1', ..., y_n')=0\\ 
	\vdots \\ 
	G_k(x, y_1, ..., y_n, y_1', ..., y_n')=0
\end{matrix}
\right. \]
$k<n$.

Уравнения $G_1,...G_k$ называются уравнениями связей. Связи называются голономными если их уравнения не зависят от производных.

Существуют функции $\lambda_1(x), ..., \lambda_k(x)$, называемые множителями \Lg, такие, что решения задачи \Lg{} являются экстремалями функционала 
\[ J=\int_{x_1}^{x_2}(F-\sum_{i=1}^{k}\lambda_i G_i)dx \] 
В случае изопериметрической задачи множители \Lg{} являются константами.
\subsection{Рассмотрите задачу о геодезических на поверхности как задачу Лагранжа. Как это позволяет охарактеризовать геодезические?}
%TODO он хочет вместе с доказательством?!

Задача отыскания геодезических в виде задачи \Lg: найти наименьшее значение интеграла
\[ I=\int_{x_1}^{x_2} \sqrt{1+y'^2+z'^2} \]
при условии, что $ G(x, y, z)=0 $.

Сведём её, по правилу множителей \Lg, к задаче на безусловный экстремум функционала
\[ J=\int_{x_1}^{x_2} \left(\sqrt{1+y'^2+z'^2}-\lambda G \right)dx \]
решая которую придём к тому, что главные нормали к геодезическим совпадают с нормалями к поверхности.

\subsection{Что такое условие трансверсальности? Каков его геометрический смысл?}

Пусть в задаче со свободными концами на экстремали интеграла
\[ I=\int_{x_1}^{x_2} F(x, y, y')dx \]
где $ x_2 $ определён как точка пересечения $ y(x) $ с заданной кривой $ \varphi(x, y)=0 $.

Уравнение %это можно назвать уравнением?
\[ \left. \left( \frac{\partial F}{\partial y'}-\frac{\frac{\partial \varphi}{\partial y}F}{\frac{\partial \varphi}{\partial x}+y'\frac{\partial \varphi}{\partial y}} \right) \right|_{x=x_2}=0 \] %в оригинале во второй дроби x и y с индексом 2. Зачем?
называется условием трансверсальности. 

Для функции $F$ вида $ F(x,y,z)=H(x,y)\sqrt{1+z^2} $ и $\varphi = g(x)-y$ (кривая свободного конца задана явно) 
условие трансверсальности упростится до вида 
\[ g'y'=-1, \]
что является условием ортогональности экстремали и кривой свободного конца.

\subsection{Сформулируйте принцип Ферма в геометрической оптике. Что представляет собой трансверсальность в геометрической оптике. Почему?}
Принцип Ферма состоит в том, что свет идёт по пути, которому соответствует наименьшее время распространения. 
Этот принцип требует минимума функционала времени 
\[ t=\int_{\gamma} \frac{ds}{v} \]  
на пути $\gamma$ светового луча. $ds$ -- элемент дуги, $v$ -- скорость света.

%TODO Что представляет собой трансверсальность в геометрической оптике. Почему?

\subsection{Выпишите первое необходимое условие. Что отличает его от уравнения Эйлера-Лагранжа?}
Первое необходимое условие
 \[ \frac{\partial F}{\partial y'} - \int_{x_1}^{x}\frac{\partial F}{\partial y}dx = \text{const}\]

%TODO Что отличает его от уравнения Эйлера-Лагранжа?

\subsection{Выпишите условие Вейерштрасса-Эрдмана (на изломе). Какое заключение оно позволяет сделать в случае функционалов геометрической оптики?}
Для каждого значения $x_0$ , соответствующего угловой точке минимизирующей кривой $y = y(x)$, правый и левый пределы функции
\[\frac{\partial F}{\partial y'} \equiv F'_{y'}(x, y, y')\]
совпадают:
\[ F'_{y'}(x_0, y(x_0), y'(x_0+0))=F'_{y'}(x_0, y(x_0), y'(x_0-0)) \]

Из этого условия, например, следует, что для задач в которых функция $ F $ имеет вид $ F=H(x, y)\sqrt{1+y'^2}$, 
минимизирующая функция не имеет угловых точек. В частности в геометрической оптике, где функционал времени 
\[ t=\int_{\gamma} \frac{ds}{v} = \int_{x_1}^{x_2}  \frac{1}{v(y)}\sqrt{1+y'^2}\] 

\section{Вариационное исчисление. Гамильтонова механика и поля экстремалей}
\subsection{Что понимается под инвариантностью интегрального функционала? Сформулируйте теорему НЁтер. Прокомментируйте с ее помощью закона сохранения энергии.}
\[ I\left[u\right]=\int_{a}^{b}\left(\vec{x},\vec{u},\nabla \vec{u}\right)dx\]
По определению функционал $I$ инвариантен относительно движения со скоростью $\vec{s}=(\vec{a},\vec{b})$ в плоскости $(\vec{x},\vec{u})$, если:
\[\frac{d}{dt}I[u_t]\equiv0\]
\[\frac{d(x,u)}{dt}=s\]
Теорема Нётер. Если $I$ инвариативен относительно движения со скоростью $\vec s=(\vec a,\vec b)$, то вдоль экстремали 
\[(-H, \vec p)\cdot \vec s = const \]
или \[\frac{d}{dt}J[\Gamma_t]=0\]
где $\Gamma_t$ -- экстремаль. 
Если учесть это в условии трансверсальности, получим: 
\[\int_{D_t} \vec n \left[F\vec{a}+(\vec b-\vec{a}\times\nabla u)\frac{\partial F}{\partial\nabla u}\right]dS=0\]
Стянув $D_t$ в точку, получим
\[\text{div}\left[F\vec{a}+(\vec b-\vec{a}\times\nabla u)\frac{\partial F}{\partial\nabla u}\right]d\vec{x}=0\]
это и есть закон сохранения энергии, ассоциированный с инвариативностью функционала $I$.  
\subsection{Опишите функцию Гамильтона как преобразование Лежандра. Что такое условие регулярности? Что такое инволютивность преобразования Лежандра? Что такое канонический вид уравнений Эйлера--Лагранжа?}
Преобразованием Лежандра называется отображение 
\[ F(x,y,z) \rightarrow H(x, y, p) = pz-F(x,y,z), \]
где 
\[ z=P(x,y,p) \: \text{и} \: p=F'_z(x, y, z), \]
при условии $\frac{\partial^2 F}{\partial z^2} \neq 0$

Применяя это преобразование дважды, получим тождественное преобразование, т.е. преобразования Лежандра инволютивно.

Преобразование Лежандра функции $ F(x, y, y') $ ведёт к функции Гамильтона
\[ H(x, y, p) = pP-F(x,y,P),\: \text{где} \: y'=P(x, y, p), \:  p=F'_{y'}(x, y, y') \]
при условии \[ \frac{\partial^2 F}{\partial y'^2} \neq 0, \]
называемом условием регулярности.

Канонический вид уравнений Эйлера--Лагранжа:
\[\left\{
\begin{matrix}
	\frac{d y}{d x}=\frac{\partial H}{\partial p}\\ 
	\frac{d p}{d x}=-\frac{\partial H}{\partial y}
\end{matrix}
\right.\]


\subsection{Дайте определение поля экстремалей. Какое поле называется центральным? Объясните, как вложение экстремали в поле помогает в определении характера этой экстремали.}

Полем экстремалей называется семейство экстермалей, явлющихся решением дифференциального уравнения 
\[ y'=z(x,y) \]
Если семейство экстремалей имеют общую начальную точку, то такое семейство -- центральное.\\
Говорят, что экстремаль $y=y(x)$ вложена в поле экстремалей, если область определения соответствующего поля направлений является окрестностью графика данной экстремали и функция $y(x)$ является решением дифференциального уравнения этого поля. \\
%TODO Объясните, как вложение экстремали в поле помогает в определении характера этой экстремали.
\subsection{Дайте определение функция поля. Какому дифференциальному уравнению она подчиняется? Объясните, как по частному решению уравнения Гамильтона--Якоби можно построить поле экстремалей?}
Функция поля определяется равенством \[S(t,q)=\int_{}^{(t,q)}[L-\langle f |\nabla_f L\rangle ]dt + \langle \nabla_f L |dq\rangle = \int_{}^{(t,q)} \langle p|dq\rangle - Hdt, \quad p = (p_1,\dots, p_2)\]
где $f$ -- функция наклона поля экстремалей. \\
Функции $p_i$ сложные
\[p_i=\frac{\partial L}{\partial\dot q_i}, \quad L=L(q_1,\dots,q_n, \dot q_1,\dots,\dot q_n), \quad \dot q_i=f_i\]
При этом \[\frac{\partial S}{\partial t} = -H, \quad \frac{\partial S}{\partial q_i} = p\]
Это приводит к уравнению Гамильтона--Якоби \[\frac{\partial S}{\partial t}+H\left(q_1,\dots,q_n,\frac{\partial S}{\partial q_1},\dots,\frac{\partial S}{\partial q_n}\right)=0\] 
или \[\frac{\partial S}{\partial t}+H(q,\nabla_qS)=0\]
Если $S(t,q_1,\dots,q_n,\alpha_1,\dots,\alpha_n)$ -- $n$-параметрическое семейство решений уравнения Гамильтона--Якоби, то в условиях регулярности и при условии 	\[\det \left( \frac{\partial^2 S}{\partial q_i\partial\alpha_j}\right)\neq0\]
решение $q=q(t)$ системы \[\frac{\partial S}{\partial\alpha_i}=\beta_i\quad(i=1,\dots,n)\] 
при любых фиксированных значениях параметров $\alpha_i$ и $\beta_i$ является экстремалью функционала $I$.
\section{Вариационное исчисление. Уравнение Якоби, сопряженные точки.}
\subsection{Дайте определение второй вариации. Что представляет собой уравнение Якоби. Что означает, что оно является уравнением в вариациях по отношению к уравнению Эйлера --Лагранжа?}
Вторая вариция интегрального функционала:
\[\delta^2 I [\eta]=\frac{d^2}{dt^2}I(y+t\eta)\bigg|_{t=0}\]
\[\delta^2 I [\eta]=\int_{x_1}^{x_2}\left[\frac{\partial^2 F}{\partial y^2}\eta^2 + 2\frac{\partial^2 F}{\partial y \partial y'}\eta\eta' + \frac{\partial^2 F}{\partial {y'}^2}{\eta'}^2\right]dx=
\int_{x_1}^{x_2} \left[A\eta^2+B\eta'^2\right]dx,\]
где
\[A=\frac{\partial^2 F}{\partial y^2} - \frac{d}{dx}\left(\frac{\partial^2 F}{\partial y \partial y'}\right) \]
\[B=\frac{\partial^2 F}{\partial y \partial y'}\]
Уравнение Якоби: 
\[A\eta-\frac{d}{dx}\left(B\eta'\right)=0 , \]
где функция $\eta$ удовлетворяет граничным условиям: 
\[\eta(x_1)=\eta(x_2)=0\]
Если уравнение $y(x,\lambda)$ удовлетворяет некоторому дифференциальному уравнению (например, $f(x,y,y',y'')=0$), то производная по параметру $\eta = y'_\lambda$ удовлетворяет уравнению
\[\frac{\partial f}{\partial y}\eta + \frac{\partial f}{\partial y'}\eta' + \frac{\partial f}{\partial y''}\eta''=0,\]
которое называется уравнением в вариациях. 
Уравнение Якоби представляет из себя уравнение в вариациях по отношению к уравнению Эйлера--Лагранжа. \\
Уравнение Э.-Л.: 
\[F_y-F_{xy'}-F_{yy'}y'-F_{y'y'}y''=0\]
Коэффициенты Якоби: 
\[A = F_{yy}-F_{xyy'}-F_{yyy'}y'-F_{y'y'}y''=0\]
\[B=F_{y'y'}\]
\[\frac{dB}{dx}=F_{xy'y'}+F_{yy'y'}y'+F_{y'y'y'}y''\]
Тогда уравнение в вариациях сводится к уравнению Якоби: 
\[A\eta-\frac{dB}{dx}\eta'-B\eta''=0\]
\subsection{Дайте определение сопряженной точки. Опишите геометрический смысл ее существования. Что такое каустика? Почему экстремаль с сопряженной точкой не может доставлять экстремум функционалу?}
Точка $(x_0,y_0)$, на экстремали $y(x)$ (т. е. $y_0=y(x_0)$) называется сопряженной с точкой $(x_1,y_1)$ (началом экстремали), если уравнение Якоби имеет решение $\eta$, обращающееся в ноль в точке $x_0$, но не равное нулю тождественно на интервале $[x_1,x_0]$.\\
Центральным называется семейство экстремалей с общей начальной точкой. Точка касания экстремали и огибающей данного центрального семейства называется точкой, сопряженной с начальной точкой экстремали. Это и сводится к геометрическому смыслу ее существования \\
Огибающая семейства лучей называется каустикой. \\
Экстремаль с сопряженной точкой не может доставлять экстремум функционалу, так как это бы противоречило необходимому условию Якоби (экстремаль удовлетворяет условию Якоби, если между началом и концом этой экстремали нет точек, сопряженных с начальной).
\section{Вариационное исчисление. Прямые методы.}
\subsection{Дайте вариационное описание собственных значений задачи Штурма--Лиувилля. Объясните рост собственных значений.}
Задача на минимум функционала 
\[I[y]=\int_{a}^{b}[p(x)y'^2(x)+q(x)y^2(x)]dx\]
на множестве дважды непрерывно дифференцируемых вещественнозначных функций $y(x)$, удовлетворяющих условиям
\[y(a)=y(b)=0\]
\[\int_{a}^{b} y^2(x)dx=1\]
Уравнение Эйлера--Лагранжа для этой изопериметрической задачи имеет вид
\[qy-(py')'-\lambda y=0\]
Тогда пара $(\lambda, y)$ является собственным значением и собственной функцией задачи Штурма--Лиувилля с условием Дирихле.
\[L[y]=qy-(py')'\]
Положим по определению 
\[\langle f|g\rangle=\int_{a}^{b}f\overline{g}dx\]
\[||f||=\sqrt{\langle y|y \rangle}\]
Связь квадратичного функционала $I[y]$ и оператора $L[y]$: 
\[I[y]=\int_{a}^{b}[py'^2 +qy^2]dx=\int_{a}^{b}[-(py')'y+qy^2]dx=\int_{a}^{b}L[y]ydx=\langle L[y]|y\rangle\]
Пусть по опрделению 
\[K(f,g)=\int_{a}^{b}[pf'g'qfg]dx\]
Этот функционал билинейный. 
Тогда
\[I[y]=K(y,y)\]
и
\[I(y+\eta)=I(y)+I(\eta)+2K(y,\eta)\]
и для функций, удовлетворяющих нулевым граничным условиям
\[K(f,g)=\int_{a}^{b}[-(pf')'+qf]gdx=\langle L[f]|g\rangle\]
верны следующие свойства:
\begin{enumerate}
	\item Если $\lambda$ -- собственное значение, отвечающее нормированной собственной функции $y$ оператора $L$, то 
	\[I[y]=\lambda\]
	\item Если $y$ собственная функция оператора $L$, а $z$ -- ортогональна к $y$, то 
	\[K(y,z)=0\]
\end{enumerate}
Все собственные значения оператора Штурма--Лиувилля $L$ могут быть расположены в возрастающую бесконечную последовательность. \\
Предположим, что все собственные значения $\lambda_1,\dots,\lambda_{n-1}$, соответствующие собственным функциям $y_1,\dots,y_{n-1}$ определены. Решение задачи на минимум функционала $I[y]$ существует и является непрерывно дифференцируемой функцией. Эта минимизирующая функция, согласно с теорией изопериметрических задач, является безусловной экстремалью функционала
\[J =\int{a}^{b}[py'^2+qy'^2-\mu y^2 -\sum_{i=1}^{n-1}\nu_iy_iy ]dx \]
где $\mu$ и $\nu_1,\dots,\nu_{n-1}$ -- множители Лагранжа. \\
Уравнение Эйлера
\[2qy-2\mu y -\sum_{i=1}^{n-1}\nu_iy_i - (py')'=0 \]
принимает вид 
\[L[y]=\mu y+\frac{1}{2}\sum_{i=1}^{n-1}\nu_iy_i\]
Умножая это равенство на $y_j$, (j<n) и интегрируя, находим
\[0=K(y,y_j)=\langle L[y]|y_j\rangle=\mu \langle y|y_j\rangle +\frac{1}{2}\sum_{i=1}^{n-1}\nu_i\langle y_i|y_j\rangle=\frac{\nu_j}{2}\]
Так как все множители $\nu_j$ равны нулю, уравнение Эйлера превращется в уравнение Штурма--Лиувилля
\[L[y]=\mu y\]
функция $y$ собственная и 
\[I[y]=\mu\]
Таким образом, $\mu$ наименьшее значение интеграла $I$. Тогда $\lambda_n\ge\mu$ и тогда $\lambda_n=\mu$. Обратившись к методу математической индукции, можно придти к выводу о росте собственных значений.
\subsection{В чем заключается минимаксное свойство собственных значений? Объясните, как это позволяет сделать вывод о росте собственных значений к бесконечности.}
Пусть $\mu$ -- минимум интеграла $I[y]$ при условиях 
\[\int_{a}^{b}y^2dx=1, \quad \int_{a}^{b} z_iydx=0 \quad (i=1,\dots, n-1)\]
где $z_1,\dots,z_{n-1}$ -- произвольные фиксированные непрерывно дифференцируемые функции. \\
Тогда \[\mu \le \lambda_n\]
где $\lambda_n$ -- собственные значения оператора Штурма--Лиувилля, ассоциированного с квадратичным функционалом $I[y]$. \\
Последовательность собственных чисел задачи Штурма--Лиувилля стремится к бесконечности: $\lambda_n \to +\infty $. \\
Положим \[p_1 = min p(x), \quad q_1 = min q(x), \quad p_2 = max p(x), q_2 = max q(x)\]
и определим функционалы \[I_j[y]=\int_{a}^{b}[p_iy'^2+q_iy^2]dx \quad (j=1,2)\]
Задачи Штурма--Лиувилля для этих функционалов \[-p_jy''+q_jy-\lambda y=0, \quad y(a)=y(b)=0\]
Решая задачу, получаем собственные значения $\lambda_n(j)$ \[\lambda_n(j)=\frac{\pi^2n^2p_j}{(b-a)^2}+q_j\]
Тогда оценка для собственных значений \[\frac{\pi^2n^2p_1}{(b-a)^2}+q_1 \le \lambda_n \ge \frac{\pi^2n^2p_2}{(b-a)^2}+q_2\]
Таким образом, $\lambda_n$ расходятся к бесконечности. 

\section{Дифференциальные уравнения}
\subsection{Что такое автономные системы дифференциальных уравнений на плоскости? Что представляет собой фазовый портрет такой системы?}
Автономная системы дифференциальных уравнений на плоскости имеет вид
\[ \left\{\begin{matrix}
	x'=f(x,y)\\ 
	y'=g(x,y)
\end{matrix}\right. \]
Решением этой системы интегральная кривая 
\[ \left\{\begin{matrix}
	x=x(t)\\ 
	y=y(t)
\end{matrix}\right. \]
А её проекция на фазовую плоскость (плоскость $ (x, y) $) называется фазовой кривой.\\
Фазовый портрет -- всё семейство фазовых кривых
%TODO проверить (не совсем понял 2ой вопрос)

\subsection{Что такое гамильтоновы системы на плоскости? Какие их свойства Вам известны?}
Гамильтонова система имеет вид 
\[ \left\{\begin{matrix}
	\dot{x}=\frac{\partial H}{\partial y}\\ 
	\dot{y}=-\frac{\partial H}{\partial x}
\end{matrix}\right. \]
где $ H=H(x,y) $

Свойства
\begin{itemize}
	\item Закон сохранения для автономной Гамильтоновой системы: $ H=\text{const} $
	\item Теорема Лиувилля: фазовый объём сохраняется %TODO надо немного пояснить
\end{itemize}


\end{document}