\documentclass{article}
\usepackage[utf8]{inputenc}
\usepackage[russian]{babel}
\usepackage[14pt]{extsizes}
\usepackage{amsmath, amsfonts, amssymb}

\begin{document}

\begin{section}{Краткие теоретические вопросы по теме "Ряды Фурье"}

\begin{subsection}{Как найти коэффициенты равномерно сходящегося тригонометрического ряда по его сумме?}
Для функции $f$ с периодом $2\pi$ представимой в виде суммы тригонометрического ряда 
\[f(x) = \frac{a_0}{2}+\sum_{n=1}^{\infty}a_n \cos nx +b_n \sin nx\]
коэффициенты $a_n$ и $b_n$ находятся по формулам
\[a_n = \frac{1}{\pi}\int_{-\pi}^\pi f(x)\cos n x dx\; (n = 0, 1, 2 ...)\]
\[b_n = \frac{1}{\pi}\int_{-\pi}^\pi f(x)\sin n x dx\; (n = 1, 2 ...)\]
\end{subsection}


\begin{subsection}{Что такое коэффициенты Фурье по ортонормированной системе в абстрактном векторном пространстве со скалярным произведением? Опишите проекционное свойство частичной суммы ряда Фурье.}
Пусть $\vec{a} \in V$, где $V$ унитарное пространство и $\{\vec{e}_n\}_{n=1}^{\infty}$ ортонормированная система.\\
Тогда коэффициентами Фурье по ортонормированной системе $\{\vec{e}_n\}$ называются числа 
\[ c_n(\vec{a}) = (\vec{a}, \vec{e_n})\]

Рассмотрим $b_n = \sum_{k=1}^n c_k(\vec{a})\vec{e_k}$. Тогда $\vec{a}-\vec{b} \perp \vec{b}$. Это и есть проекционное свойство.
\end{subsection}


\begin{subsection}{Что такое равенство Парсеваля? Объясните, почему коэффициенты Фурье обязаны убывать.}
Из неравенства Бернулли 
\[\sum_{i=1}^{\infty}|c_i(\vec{a})|^2\leq ||\vec{a_n}||^2\]
по необходимому признаку сходимости ряда следует, что $c_i(\vec{a}) \rightarrow 0$ при $n \rightarrow 0$.
Равенство 
\[\sum_{i=1}^{\infty}|c_i(\vec{a})|^2\ = ||\vec{a_n}||^2\]
называется уравнением замкнутости или равенством Парсеваля
\end{subsection}


\begin{subsection}{В чем состоит минимизирующее свойство коэффициентов Фурье?}
Пусть $a \in V$ и $\{\vec{e}_n\}_{n=1}^{n}$ --- произвольная ортонормированная система.\\
Тогда \[\min_{\lambda_1,...,\lambda_n} \Delta (\lambda_1,...,\lambda_n) = \Delta(c_1,...,c_n),\]
где $\Delta (\lambda_1,...,\lambda_n) = ||\vec{a}-\sum_{k=1}^n\lambda_k\vec{e_k}||$ и 
$c_k = c_k(\vec{a})$ --- коэффициенты Фурье относительно ортонормированной системы $\{\vec{e}_n\}_{n=1}^{n}$
\end{subsection}


\begin{subsection}{Что такое ряд Фурье на пространстве\\$2l$-периодических функций}
Зададим в пространстве непрерывных $2l$-периодических функций скалярное произведение
\[(f, g) = \frac{1}{2l}\int_0^{2l}f(x)\overline{g(x)}dx,\]
тем самым превратив его в унитарное. Обозначим его как $C_{2l}$.\\

$e_n: x \mapsto e^{i\frac{\pi}{l}nx}, n \in \mathbb{Z}$, $e_n$ образуют ортонормированную систему
Коэффициенты Фурье относительно этой ортонормированной системы
\[c_n(f) = \frac{1}{2l} \int_0^{2l}f(x)e^{-i\frac{\pi}{l}nx}dx\]
Ряд Фурье 
\[f(x) = \sum_{n \in \mathbb{Z}} c_n(f)e^{i\frac{\pi}{l}nx}\]
В предыдущем выражении равенство в смысле среднеквадратичной нормы.
\end{subsection}


\begin{subsection}{Как разложить функцию, заданную на интервале $(0, l)$, в ряд по косинусам кратных дуг? по синусам?}
Воспользуемся в этом вопросе вещественной формой ряда Фурье
\[f(x) \sim \frac{a_0}{2}+\sum_{n=1}^{\infty}a_n \cos \frac{\pi nx}{l} +b_n \sin \frac{\pi nx}{l}\]
и коэффициенты запишем в симметричной форме
\[a_n = \frac{1}{\l}\int_{-l}^l f(x)\cos \frac{\pi nx}{l} dx\; (n = 0, 1, 2 ...)\]
\[b_n = \frac{1}{\l}\int_{-l}^l f(x)\sin \frac{\pi nx}{l} dx\; (n = 1, 2 ...)\]

Далее для нахождения разложения на интервале $(0, l)$ можно продолжить функцию периодически на всю ось с периодом $T = l$ тогда получится ряд содержащий и синусы, и косинусы. Если же предварительно продолжить функцию чЁтно(нечЁтно) относительно нуля на интервал $ (-l, l) $ и затем периодически $T = 2\l$, то в ряду останутся слагаемые только с косинусами(синусами), так как при вычислении коэффициента $ b_n $ ($a_n$) будет интегрироваться нечЁтная функция по симметричному интервалу.

Аналогично для получения разложения косинусам (синусам) кратных дуг нужно перед чЁтным(нечЁтным) продолжением относительно 0 продолжить функцию чЁтно/нечЁтно (в зависимости от того какие кратности дуг нужны (см. таблицу ниже)) относительно $l$.

\begin{table}[h]
	\begin{tabular}{|l|l|l|}
		\hline
		Продолж. относ. $0$     & Продолж. относ. $l$ 	& Тип разложения        \\ \hline
		ЧЁтно                        & ЧЁтно                        & Косинусы чЁтных дуг   \\ \hline
		ЧЁтно                        & НечЁтно                      & Косинусы нечЁтных дуг \\ \hline
		НечЁтно                      & ЧЁтно                        & Синусы чЁтных дуг     \\ \hline
		НечЁтно                      & НечЁтно                      & Синусы нечЁтных дуг   \\ \hline
	\end{tabular}
\end{table}

\end{subsection}


\begin{subsection}{Дайте определение свертки периодических функций. Как найти коэффициенты Фурье свертки?}
Пусть $f$ и $g$ непрерывные $2l$-периодические функции. СвЁрткой $f*g$ называется 
\[f*g(x) = \frac{1}{2l}\int_0^{2l}f(t)g(x-t)dt\]
Свойства $f*g$
\begin{itemize}
	\item
	$f*g$ непрерывная $2l$-периодическая функция
	\item
	если $g$ дополнительно k раз непрерывно дифференцируема, то $f*g$ --- тоже и 
	\[(f*g)^{(k)} = f*g^{(k)}\]
	\item
	$f*g$ билинейна, комутативна, ассоциативна
\end{itemize}

Коэффициенты Фурье свертки находятся как произведение коэффициентов Фурье каждой из функций, то есть
\[c_n(f*g)=c_n(f)c_n(g)\]
\end{subsection}


\begin{subsection}{Как найти производную свертки, если один из сверточных сомножителей дифференцируем? Как это свойство можно использовать для сглаживания функции?}
Пусть $f$ и $g$ непрерывные $2l$-периодические функции. СвЁрткой $f*g$ называется 
\[f*g(x) = \frac{1}{2l}\int_0^{2l}f(t)g(x-t)dt\]
Eсли $g$ дополнительно k раз непрерывно дифференцируема, то $f*g$ --- тоже и 
\[(f*g)^{(k)} = f*g^{(k)}\]

Множество функций $C_{2l}^1$ плотно в $ C_{2l} $, то есть 
\[ \forall f(x) \in C_{2l} \ \text{и} \ \forall \varepsilon > 0 \  \exists g(x) \in C_{2l}^1: \max_{x \in [0, 2l]}|f(x)-g(x)| < \varepsilon\]
Задача нахождения сглаженной функции состоит в получении $g(x)$.\\
ВозьмЁм сглаживающую функцию $\omega(x)$ обладающую следующими свойствами:
\begin{itemize}
	\item
	$\omega(x) \in C_{2l}^1$
	\item
	$\omega(x) \geq 0$
	\item
	$ \omega(-x)=\omega(x) $
	\item
	$ \omega(-x)=0, \forall x \in [\delta, 2l] $
	\item
	$\frac{1}{2l}\int_0^{2l}\omega(x)dx = 1$
\end{itemize}
Тогда в качестве $g(x)$ можно использовать свЁртку $f*\omega$, что следует из свойства производной свЁртки 
\end{subsection}


\begin{subsection}{Что такое фильтр и передаточная функция в теории обработки радиосигналов? Объясните, почему не существует идеального фильтра.}
Отображение $f \rightarrow f*g$ описывает прохождение сигнала $f$ через фильтр $g$. 
В результате амплитуда $c_n(f)$ n-ой гармоники $f$ умножается на $c_n(g)$.

В силу леммы Римана-Лебега $c_n(g)\rightarrow 0, n\rightarrow \infty$, значит $\nexists g: f*g=f$ (не существует фильтра не искажающего сигнал).

Передаточной называется функция отношения выходного сигнала к входному, то есть передаточная функция $ W(x) = \frac{f*g(x)}{f(x)} $ 
\end{subsection}


\begin{subsection}{Что утверждает лемма Римана-Лебега? Какова ее связь со стремлением коэффициентов Фурье к нулю?}

Лемма Римана-Лебега состоит в том, что если $f(x)\in C[a,b]$, то
\[\int_a^b f(x)e^{ikx}dx \rightarrow 0, k \rightarrow 0\]
Заметим, что при $a = 0$, $b = 2l$, $k=-\frac{\pi}{l}n, n \in \mathbb{Z}$ и $f(x)\in C_{2l}$ 
\[\int_a^b f(x)e^{ikx}dx \rightarrow 0 = 2l \frac{1}{2l}\int_0^{2l}f(x)e^{-i\frac{\pi}{l}nx}dx = 2lc_n(f) \rightarrow 0, k \rightarrow 0\]
Значит $c_n(f) \rightarrow 0, k \rightarrow 0$, 
где $c_n$ --- коэффициенты Фурье относительно ортонормированной системы ${e_n}$, $e_n: x \mapsto e^{i\frac{\pi}{l}nx}, n \in \mathbb{Z}$
\end{subsection}


\begin{subsection}{Сформулируйте теорему Дирихле для непрерывно дифференцируемых функций. Почему ряд Фурье сходится к такой функции равномерно?}
Теорема Дирихле состоит в том, что если функция $f \in C_{2l}^1$, то ряд Фурье сходится к $f$ поточечно
\[f(x)=\sum_{n \in \mathbb{Z}} c_n e^{i\frac{\pi}{l}nx}, \]
где $x \in \mathbb{R}$ и $c_n = \frac{1}{2l}\int_0^{2l}f(x)e^{-i\frac{\pi}{l}nx}dx$

Так как
\[ c_n(f) = \frac{c_n(f')}{i\frac{\pi}{l}n} \]
и \[ \left| \frac{c_n(f')}{n} \right| \leq \left|\frac{|c_n(f')|^2 + \frac{1}{n^2}}{2}\right|\]
В предыдущем неравенстве было использовано, что среднее геометрическое меньше среднего квадратичного.
\[ |c_n(f)| \leq \left| \frac{c_n(f')}{n} \right| \leq \left|\frac{|c_n(f')|^2 + \frac{1}{n^2}}{2}\right|\] 
Ряд $\sum |c_n(f')|^2$ сходится в силу неравенства Бесселя, $\sum 1/n^2$ сходится. Значит сходится и $\sum |c_n(f)|$ 
\end{subsection}


\begin{subsection}{Что такое сходимость рядов Фурье в среднеквадратичном? Какова связь такой сходимости с равенством Парсеваля?}
Если функция f непрерывна и периодична с периодом $ 2l $, то ряд Фурье сходится к функции $ f $ в среднеквадратичном
\[ ||f-\sum_{k=-n}^{n} c_k e_k|| \underset{n \rightarrow \infty}{\longrightarrow} 0 ,\] 
где $c_n = \frac{1}{2l}\int_0^{2l}f(x)e^{-i\frac{\pi}{l}nx}dx$ и $e_n = e^{i\frac{\pi}{l}nx}$

Воспользуемся минимизирующим свойством коэффициентов Фурье 
\[ ||f-\sum_{k=-n}^{n} c_k e_k|| = 
||f|| - \sum_{k=-n}^{n}c_k \underset{n \rightarrow \infty}{\longrightarrow} 0\]
откуда получим равенство Парсеваля
\[ ||f|| = \sum_{k \in \mathbb{Z}} c_k \]
\end{subsection}


\begin{subsection}{Как найти коэффициенты Фурье первообразной функции с нулевым средним?}
Пусть $ f $ ---  $ 2\pi $ переодическая функция и $ c_0 = 0 $ (среднее значение $f$ равно нулю), тогда первообразная $ F(x) = \int_{0}^{x}f(t)dt $ тоже $ 2\pi $ периодическая функция и 
\[ F(x) = \sum_{n \neq 0}c_n \int_{0}^{x} e^{int}dt \]
\end{subsection}


\begin{subsection}{Что представляет собой ряд Фурье производной? Какова связь между гладкостью функции и скоростью убывания коэффициентов Фурье?}
Для $ 2\pi $ периодической $ k $ раз непрерывно дифференцируемой функции $ f \in C_{2\pi}^{(k)} $
\[ c_n(f^{k}) = (in)^k c_n(f)  \]

Чем глаже функция, тем быстрее убывают коэффициенты 
\[ c_n(f) = o\left(\frac{1}{n^k}\right)\]
Обратное тоже верно, но в другой формулировке:\\
Если
\[ c_n = \frac{\sigma_n}{n^{k+1}}, \]
где \[ \{\sigma_n\}: \sum |\sigma_n|^2 - \text{сходится}  \]
то $f \in C_{2\pi}^k$
\end{subsection}

\end{section}

%----------------------------------

\begin{section}{Задача Штурма-Лиувилля?}


\begin{subsection}{Как ставится регулярная задача Штурма-Лиувилля? Что такое собственные функции и собственные значения такой задачи? Опишите основные свойства системы собственных функций и собственных значений задачи Штурма-Лиувилля.}
Поставим задачу отыскать на отрезке $ [a, b] $ решения $ y = y(x)$ дифференциального уравнения 
\[ p_2(x)y'' + p_1(x)y' + p_0(x)y = f(x), \]
удовлетворяющее краевым условиям 
\[ \left\{\begin{matrix}
	\alpha_0 y(a) + \alpha_1 y'(a) = c_1\\ 
	\beta_0 y(b) + \beta_1 y'(b) = c_2
\end{matrix}\right. \]

Пусть $ p_0, p_1, p_2, f $ --- непрерывные и краевые условия однородны ($ с_1 = с_2 = 0 $)

ВведЁм оператор $ L $ действующий из пространства $V_2$ дважды непрерывно дифференцируемых функций на $ [a, b] $, удовлетворяющих краевым условиям 
\[ L(y) = p_2(x)y'' + p_1(x)y' + p_0(x)y \]
Тогда краевая задача сведЁтся к нахождению таких $ y \in V_2 $, что $ L(y) = f $

Пусть $ \rho $: $ p_1 \rho = (p_2 \rho)'$ и $p = -p_2 \rho$. Домножим и разделим левую часть дифференциального уравнения на $\rho$. Тогда оператор  
\[ L(y) = \frac{-(py')' + qy}{\rho} \]
называется оператором Штурма-Лиувилля. 

Краевая задача на собственные числа и собственные функции оператора $L$
\[ -(py')' + qy = \lambda \rho y \]
\[ \left\{\begin{matrix}
	\alpha_0 y(a) + \alpha_1 y'(a) = c_1\\ 
	\beta_0 y(b) + \beta_1 y'(b) = c_2
\end{matrix}\right. \]
называется задачей Штурма-Лиувилля, при этом предполагается, что $ p $, $ q $ и $ \rho $ --- вещественные непрерывные функции, \textit{(здесь должны быть ещЁ условия наложенные на $ p $, $ q $ и $ \rho $:  причЁм $ p $ --- непрерывно дифференцируема ...) }

Задача Штурма-Лиувилля --- регулярная, если $L$ --- регулярный оператор, то есть $p, \rho > 0$

Свойства решений задачи Штурма-Лиувилля
\begin{itemize}
	\item
	Корни собственных функций просты. (если $y$ --- собственная функция, то еЁ нули простые)
	\item
	Каждому собственному значению отвечает единственная с точностью до множителя собственная функция (т.е. собственные числа оператора Штурма–Лиувилля --- простые)
	\item
	Собственные значения задачи Штурма–Лиувилля вещественны. Соответствующие им собственные функции могут быть выбраны вещественными.
	\item
	Различным собственным значениям $ \lambda_1 $ и $ \lambda_2 $ отвечают ортогональные собственные функции $ y_1 $ и $ y_2 $
	\item
	Собственные числа образуют бесконечную монотонно возрастающую последовательность, стремящуюся к бесконечности
\end{itemize}
\end{subsection}

\begin{subsection}{В чем состоит метод Фурье для задачи о колебаниях неоднородной струны?}
Рассмотрим струну, закреплЁнную в точках $ 0 $ и $ \pi $ на оси $ x $ с положением равновесия по отрезку $ [0, \pi] $\\
Уравнение для функции $ u(x, t) $, описывающее форму струны в момент времени $ t $ (волновое уравнение)
\[ \frac{\partial^2u}{\partial t^2} = a^2 \frac{\partial^2u}{\partial x^2}\]

Зададим начальные условия 
\[ \left\{\begin{matrix}
	u(x, 0) = f(x)\\ 
	\left. \frac{\partial u}{\partial t} \right| _{t=0} = 0
\end{matrix}\right. \]

Будем искать решение в виде ряда Фурье по синусам
\[ u(x, t) = \sum_{n=1}^{\infty} b_n(t)\sin(nx) \]

Подставив $ u(x, t) $ в волновое уравнение и начальные условия получим, что
\[ b_n '' (t) = -a^2 n^2 b_n(t) \]
\[ \left\{\begin{matrix}
	b_n(0) = b_n\\ 
	b_n'(0) = 0
\end{matrix}\right. \]
где $b_n = \frac{2}{\pi} \int_{0}^{\pi}f(x) \sin(nx)dx$.\\
Откуда получаем, что
\[ b_n(t)=b_n cos(ant) \]
и 
\[ u(x, t) = \sum_{n=1}^{\infty} =b_n \cos(ant)\sin(nx) \]

Однако надо дополнительно потребовать, чтобы полученный ряд был дважды непрерывно дифференцируемым. Если $ f(x) \in C_{[0, \pi]}^3$ и $ f''(0)=f''(\pi)=0 $, то ряд $ \sum_{n=1}^{\infty} |b_n|n^2 $ и дополнительное условие выполняется.\\
В этом случае 
\[ b_n = \frac{2}{\pi n^3} \int_{0}^{\pi}f'''(x) \cos(nx)dx \]

Разложив $  \cos(ant)\sin(nx) $ по формуле произведения синуса на косинус и воспользовавшись тем, что $ f(x) = \sum_{n=1}^{\infty} b_n \sin(nx) $ получим, что
\[ u(x, t) = \frac{f(x-at) + f(x+at)}{2} \]
\end{subsection}

\end{section}

\end{document}
