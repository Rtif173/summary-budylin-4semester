\documentclass{article}
\usepackage[utf8]{inputenc}
\usepackage[russian]{babel}
\usepackage[14pt]{extsizes}
\usepackage{amsmath, amsfonts, amssymb}

\begin{document}


\begin{section}{Краткие теоретические вопросы по теме "Ряды Фурье"}

\begin{subsection}{Как найти коэффициенты равномерно сходящегося тригонометрического ряда по его сумме?}
Для функции $f$ с периодом $2\pi$ представимой в виде суммы тригонометрического ряда 
\[f(x) = \frac{a_0}{2}+\sum_{n=1}^{\infty}a_n \cos nx +b_n \sin nx\]
коэффициенты $a_n$ и $b_n$ находятся по формулам
\[a_n = \frac{1}{\pi}\int_{-\pi}^\pi f(x)\cos n x dx\; (n = 0, 1, 2 ...)\]
\[b_n = \frac{1}{\pi}\int_{-\pi}^\pi f(x)\sin n x dx\; (n = 1, 2 ...)\]
\end{subsection}


\begin{subsection}{Что такое коэффициенты Фурье по ортонормированной системе в абстрактном векторном пространстве со скалярным произведением? Опишите проекционное свойство частичной суммы ряда Фурье.}
Пусть $\vec{a} \in V$, где $V$ унитарное пространство и $\{\vec{e}_n\}_{n=1}^{\infty}$ ортонормированная система.\\
Тогда коэффициентами Фурье по ортонормированной системе $\{\vec{e}_n\}$ называются числа 
\[ c_n(\vec{a}) = (\vec{a}, \vec{e_n})\]

Рассмотрим $b_n = \sum_{k=1}^n c_k(\vec{a})\vec{e_k}$. Тогда $\vec{a}-\vec{b} \perp \vec{b}$. Это и есть проекционное свойство.
\end{subsection}


\begin{subsection}{Что такое равенство Парсеваля? Объясните, почему коэффициенты Фурье обязаны убывать.}
Из неравенства Бернулли 
\[\sum_{i=1}^{\infty}|c_i(\vec{a})|^2\leq ||\vec{a_n}||^2\]
по необходимому признаку сходимости ряда следует, что $c_i(\vec{a}) \rightarrow 0$ при $n \rightarrow 0$.
Равенство 
\[\sum_{i=1}^{\infty}|c_i(\vec{a})|^2\ = ||\vec{a_n}||^2\]
называется уравнением замкнутости или равенством Парсеваля
\end{subsection}


\begin{subsection}{В чем состоит минимизирующее свойство коэффициентов Фурье?}
Пусть $a \in V$ и $\{\vec{e}_n\}_{n=1}^{n}$ --- произвольная ортонормированная система.\\
Тогда \[\min_{\lambda_1,...,\lambda_n} \Delta (\lambda_1,...,\lambda_n) = \Delta(c_1,...,c_n),\]
где $\Delta (\lambda_1,...,\lambda_n) = ||\vec{a}-\sum_{k=1}^n\lambda_k\vec{e_k}||$ и 
$c_k = c_k(\vec{a})$ --- коэффициенты Фурье относительно ортонормированной системы $\{\vec{e}_n\}_{n=1}^{n}$
\end{subsection}


\begin{subsection}{Что такое ряд Фурье на пространстве\\$2l$-периодических функций}
Зададим в пространстве непрерывных $2l$-периодических функций скалярное произведение
\[(f, g) = \frac{1}{2l}\int_0^{2l}f(x)\overline{g(x)}dx,\]
тем самым превратив его в унитарное. Обозначим его как $C_{2l}$.\\

$e_n: x \mapsto e^{inx}, n \in \mathbb{Z}$, $e_n$ образуют ортонормированную систему
Коэффициенты фурье относительно этой ортонормираванной системы
\[c_n(f) = 1/2l \int_0^{2l}f(x)e^{-inx}dx\]
Ряд Фурье 
\[f(x) = \sum_{n \in \mathbb{Z}} c_n(f)e^{inx}\]
В предыдущем выражении равенство в смысле среднеквадратичной нормы.
\end{subsection}


\begin{subsection}{Как разложить функцию, заданную на интервале $(0, l)$, в ряд по косинусам кратных дуг? по синусам?(Пустой)}
\end{subsection}


\begin{subsection}{Дайте определение свертки периодических функций. Как найти коэффициенты Фурье свертки?}
Пусть $f$ и $g$ непрерывные $2l$-периодические функции. СвЁрткой $f*g$ называется 
\[f*g(x) = \frac{1}{2l}\int_0^{2l}f(t)g(x-t)dt\]
Свойства $f*g$
\begin{itemize}
	\item
	$f*g$ непрерывная $2l$-периодическая функция
	\item
	если $g$ дополнительно k раз непрерывно дифференцируема, то $f*g$ --- тоже и 
	\[(f*g)^{(k)} = f*g^{(k)}\]
	\item
	$f*g$ билинейна, комутативна, ассоциативна
\end{itemize}

Коэффициенты Фурье свертки находятся как произведение коэффициентов Фурье каждой из функций, то есть
\[c_n(f*g)=c_n(f)c_n(g)\]
\end{subsection}


\begin{subsection}{Как найти производную свертки, если один из сверточных сомножителей дифференцируем? Как это свойство можно использовать для сглаживания функции?(Не полный)}
Пусть $f$ и $g$ непрерывные $2l$-периодические функции. СвЁрткой $f*g$ называется 
\[f*g(x) = \frac{1}{2l}\int_0^{2l}f(t)g(x-t)dt\]
Eсли $g$ дополнительно k раз непрерывно дифференцируема, то $f*g$ --- тоже и 
\[(f*g)^{(k)} = f*g^{(k)}\]
\end{subsection}


\begin{subsection}{Что такое фильтр и передаточная функция в теории обработки радиосигналов? Объясните, почему не существует идеального фильтра.}
Отображение $f \rightarrow f*g$ описывает прохождение сигнала $f$ через фильтр $g$. 
В результате амплитуда $c_n(f)$ n-ой гармоники $f$ умножается на $c_n(g)$
В силу леммы Римана-Лебега $c_n(g)\rightarrow 0, n\rightarrow \infty$, значит $\nexists g: f*g=f$ (не существует фильтра не искажающего сигнал).
\end{subsection}


\begin{subsection}{Что утверждает лемма Римана-Лебега? Какова ее связь со стремлением коэффициентов Фурье к нулю?}
\end{subsection}
Лемма Римана-Лебега состоит в том, что если $f(x)\in C[a,b]$, то
\[\int_a^b f(x)e^{ikx}dx \rightarrow 0, k \rightarrow 0\]
Заметим, что при $a = 0$, $b = 2l$, $k=-n, n \in \mathbb{Z}$ и $f(x)\in C_{2l}$ 
\[\int_a^b f(x)e^{ikx}dx \rightarrow 0 = 2l \frac{1}{2l}\int_0^{2l}f(x)e^{-inx}dx = 2lc_n(f) \rightarrow 0, k \rightarrow 0\]
Значит $c_n(f) \rightarrow 0, k \rightarrow 0$, 
где $c_n$ --- коэффициенты Фурье относительно ортонормированной системы ${e_n}$, $e_n: x \mapsto e^{inx}, n \in \mathbb{Z}$
\end{section}


\begin{subsection}{Сформулируйте теорему Дирихле для непрерывно дифференцируемых функций. Почему ряд Фурье сходится к такой функции равномерно?(Не полностью)}
Теорема Дирихле состоит в том, что если функция $f \in C_{2l}^1$, то ряд Фурье сходится к $f$ поточечно
\[f(x)=\sum_{n \in \mathbb{Z}} c_n e^{inx}, \]
где $x \in \mathbb{R}$ и $c_n = \frac{1}{2l}\int_0^{2l}f(x)e^{-inx}dx$


\end{subsection}


\end{document}
